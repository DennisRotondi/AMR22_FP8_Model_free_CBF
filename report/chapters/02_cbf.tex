
% TODOES:
%section 1: CBF inserendo le defnizioni e risultati più importanti dei papers, descrivendo il dualismo con CLF che poi verrà referenziato nella sezione MFCBF per la dimostrazione (Exponential stability sufficiente).


% In this project, we will address robotic systems with the following dynamic model:
% \begin{equation}
%     \Dm(\qv)\ddqv+\Cm(\qv,\dqv)\dqv+\Gm(\qv) = \Bm\uv
%     \label{eq:consdynmod}
% \end{equation}
% % \begin{equation}
% %     \Dm(\qv)\ddqv+\nv(\qv,\dqv) = \Bm(\qv)\uv+\Am(\qv)\lambdav
% %     \label{eq:consdynmod}
% % \end{equation}
% % \begin{equation}
% %     \Am^T(\qv)\dqv = \zerov
% %     \label{eq:const}
% % \end{equation}
% Where $\qv$ is a vector of generalized coordinates that are part of the configuration space $\calCm$ of dimension $N$; $\Dm(\qv) \in \mathbb{R}^{N\times N}$ is the inertia matrix, $\Cm(\qv,\dqv) \in \mathbb{R}^{N\times N}$ collects the centrifugal and Coriolis forces, $\Gm(\qv) \in \mathbb{R}^{N}$ involves gravity terms and $\Bm \in \mathbb{R}^{N\times M}$ is the input matrix

% $\Dm(\qv)$ is the inertia matrix; $\nv(\qv,\dqv)$ contains gravity terms, centrifugal and Coriolis forces; $\Bm(\qv)$ is the input matrix and $\Am(\qv)$ is the transpose of the matrix characterizingthe kinematic constraints that together with the Langrange multipliers $\lambdav$ reprsents the vector of reaction forces at generalized coordinate level. Dimensions and many other details are found in \cite{robbook}. In case of unconstrained robot simply $\Am(\qv) = \zerom$ making relevant only \eqref() , 

\section{Safety-Critical Control for Robotic Systems}\label{sec:cbf}
A safety controller is not intended to function independently, but rather to support the primary, potentially unsafe controller that is designed to achieve standard control objectives such as stability, regulation, or tracking. To guarantee full comprehension of the tools used hereinafter, we will briefly review several concepts and provide the necessary definitions. 
Firstly, the robotic systems under examination are those in the nonlinear affine form:
\begin{equation}
    \dot{\xv}=\fv(\xv)+\gv(\xv)\uv,
    \label{eq:system}
\end{equation}
where $\xv\in X\subseteq \mathbb{R}^n$ is the state, $\uv\in U \subseteq \mathbb{R}^m$ the control input, with $\fv : X\to\mathbb{R}^n$ and $\gv : X\to\mathbb{R}^{n\times m} $ locally Lipschitz continuous functions. 
%We also assume that $x^* = 0$ is an equilibrium of \eqref{eq:system} and X is an open and connected neighborhood of $x^*$.
%We say that a continuous function $f: [0,b) \rightarrow \mathbb{R}_{\geq 0}, where $b \in \mathbb{R}_{\geq 0}$ (or $f: (a,b) \rightarrow R$, where $a,b \in R^+$) is of class-K (or extended class-K) if it is strictly monotonically increasing and $f(0) = 0$. need for dist
\subsection{Control Lyapunov Functions}
In Lyapunov theory, stabilizing \eqref{eq:system} to a point $\xv^*=\zerov$ (assuming w.l.o.g.
$\xv^*$ is an equilibrium and 
$X$ is an open and connected neighborhood of $\xv^*$) can be achieved by finding a feedback control law $\uv=\kv(\xv)$ that drives a positive definite function, $V: D \subset \mathbb{R}^n \to \mathbb{R}_{\geq 0}$, to zero. That is, if:
\begin{equation}
    \forall \xv\in X ~~~~ \inf_{\uv\in U}[\dot{V}(\xv,\uv)] \leq -\gamma(V(\xv)),
\label{eq:condition}
\end{equation}
where
\begin{equation}
    \dot{V}(\xv,\kv(\xv)) = \nabla V(\xv)\cdot(\fv(\xv)+\gv(\xv)\uv)= L_{\fv}V(\xv)+L_{\gv}V(\xv)\uv.
\label{eq:vdot}
\end{equation}
Note: $\gamma : \mathbb{R}_{\geq 0} \to \mathbb{R}_{\geq 0} $ is a class-$\mathcal{K}$ function here, 
recalling that a continuous function $\zeta : [0, b) \to \mathbb{R}_{\geq 0},~b \in \mathbb{R}_{\geq 0}$ is of class-$\mathcal{K}$ (or $\zeta : [-a, b) \to \mathbb{R}_{\geq 0},~a,b \in \mathbb{R}_{\geq 0}$ is of extended class-$\mathcal{K}$) if it is strictly monotonically increasing and $\zeta(0) = 0$. We say to have an extended class-$\mathcal{K}_\infty$ if $(a,b) = (-\infty, \infty) = \mathbb{R}$, the entire real line.
\begin{theorem}\label{th:clft}
If $V$ is a control Lyapunov function (CLF) for \eqref{eq:system}, i.e. a positive definite function satisfing \eqref{eq:condition}, then any Lipschitz continuous feedback controller $\uv(\xv$) satisfying $\dot{V}(\xv,\uv) \leq -\gamma(V(\xv))~\forall \xv\in X$ asymptotically stabilizes the system to $\xv^*$.
\end{theorem}
\subsubsection{Exponential Stability}
Another very important stability property can be introduced from the following:
\begin{definition}
If there exist $c,k_1,k_2,\lambda \in \mathbb{R}_{> 0} \text{~such that~} \forall \xv \in X:$ 
\begin{equation}
    \begin{split}
    k_1||\xv||^c \leq V(\xv) \leq k_2||\xv||^c,  \\
    \inf_{{\uv\in \mathbb{R}^m}}[\dot{V}(\xv,\uv)] \leq -\lambda V(\xv),~
    \end{split}
    \label{eq:expstabcond}
\end{equation}
for a continuously differentiable function $V : X \to \mathbb{R}_{\geq 0}$, then $V$ is a CLF for \eqref{eq:system}.
\label{def:expclf}
\end{definition}

\begin{theorem}\label{th:clftexp}
If $V$ is a CLF that respects Definition \ref{def:expclf}, then any locally Lipschitz continuous controller $\uv=\kv(\xv)$ satisfying 
\begin{equation}
\dot{V}(\xv,\uv) \leq -\lambda V(\xv),
\end{equation}
$\forall \xv \in X$ renders $\xv^*$ exponentially stable i.e.~there exist $a,A,\beta \in \mathbb{R}_{>0}$ such that 
$|| \xv_0 || \leq a \Rightarrow || \xv(t) || \leq Ae^{-\beta t}|| \xv_0 ||,~\forall t\geq 0.$
\end{theorem}

\subsection{Control Barrier Functions}

In contrast to stability, which is concerned with guiding a system towards a specific point or set, safety can be conceptualized as enforcing the invariance of a set. For a dynamic system, safety constraints can be viewed as defining a safe region within its state space
where the state must always reside. This region is referred to as the safe set $\mathcal{S}$. 

\begin{definition} Let $\mathcal{S}\subset X\subseteq \mathbb{R}^n$ be a compact set that is the $0$-superlevel set of a continuously  differentiable function $h:X\to\mathbb{R}$, $\forall \xv \in X$
    \begin{equation}
        \begin{aligned}
\mathcal{S} =& \{\xv\in\mathbb{R}^n :& h(\xv)\geq 0\}, \\
\partial{\mathcal{S}} =& \{\xv\in\mathbb{R}^n :& h(\xv)= 0\}, \\
\mathrm{Int}({\mathcal{S}}) =& \{\xv\in\mathbb{R}^n :& h(\xv)> 0\}. \\
\end{aligned}
\end{equation}
\end{definition}
\noindent
A matematical tool as a means to guarantee the invariance property of set $\mathcal{S}$ for \eqref{eq:system} is represented by the use of Control Barrier Functions (CBF).

\begin{definition} System \eqref{eq:system} is \textit{safe} w.r.t. $\mathcal{S}$, if $\mathcal{S}$ is forward invariant under \eqref{eq:system}, that is, $\xv_0\in\mathcal{S}\Rightarrow \xv(t)\in\mathcal{S}, \forall t\geq0$.
Then, $h$ is a CBF for \eqref{eq:system} if $\frac{\partial{h}}{\partial \xv} (\xv)\neq 0$ for all $\xv\in \partial{\mathcal{S}}$ and there exists $\alpha \in \mathbb{R}_{>0}$ such that $\forall \xv \in \mathcal{S}$
\begin{equation}
\sup_{\uv\in U} \underbrace{ 
\begin{bmatrix}
L_{\fv}h(\xv) + L_{\gv}h(\xv)\uv] 
\end{bmatrix}}_{\dot{h}(\xv,\uv)}\geq -\alpha h(\xv).
\end{equation}


\begin{theorem}
    \label{th:cbf}If $h : X \to \mathbb{R}$ is a CBF for \eqref{eq:system}, then any locally Lipschitz continuous controller $\uv = \kv(\xv)$ satisfying
\begin{equation}
\dot{h}(\xv,\kv(\xv))\geq -\alpha h(\xv),
\label{cond:2}
\end{equation}
$\forall{\xv}\in\mathcal{S}$ renders \eqref{eq:system} safe with respect to $\mathcal{S}$. Additionally, the set $\mathcal{S}$ is asymptotically stable in $X$, hence with the beneficial of drive the system back to the set $\mathcal{S}$ in case of noise or modeling errors.
\end{theorem}
\end{definition}
\noindent
Generically for these results $\alpha h(\xv)$ may be taken as any extended class-$\mathcal{K}_{\infty}$ function $\alpha(\cdot)$ of $h$.
It interesting to point out that CLFs yield invariant level sets as well. If these level sets are contained in the safe set one can guarantee safety, however clearly they are overly strong and conservative. On the other hand, the action of CBFs can be modulated by $\alpha(\cdot)$ depending on whether a more conservative or aggressive behavior is desired, while allowing the state to evolve freely within the safe set $\mathcal{S}$. Ideally, as $\alpha \to \infty$, the CBF would activate only on the boundary of the set.


\subsubsection{Control synthesis}
Theorem \ref{th:cbf} establishes safety-critical controller 
synthesis by means of condition (\ref{cond:2}). However, satisfying  it may require a complex analytic procedure. A minimally invasive strategy to obtain this result is to set up an optimization-based control problem whose objectives is to minimize the difference between the desired, possibly unsafe controller $\kv_d(\xv)$ and the actuated one $\uv$, so as to satisfy the CBF constraint
\begin{equation}\label{eq:optprob}
\begin{aligned}
\kv(\xv)=\underset{\uv\in U}{\arg\min}  &(\uv-\kv_d(\xv))^T(\uv-\kv_d(\xv)), \\ 
\mathrm{s.t.}\,\, &\dot{h}(\xv,\uv)\geq -\alpha h(\xv).
\end{aligned}
\end{equation}
Additional requirements can be imposed and it can be shown that the controller obtained through this method is Lipschitz continuous.
\subsection{Effect of disturbances}
In reality, all robotics systems are typically subject to unknown disturbances that may affect and so compromise the stability and safety properties. For instance, a bounded disturbance $\dv\in\mathbb{R}^m$ added to the input $\uv$ leads to the system $\dot{\xv} = \fv(\xv)+\gv(\xv)(\uv+\dv)$. To deal with these disturbances, the notion of exponential \textit{input-to-state stability} (ISS) is of paramount importance. It comes by extending the definition of exponential stability by requiring the existency of a class-$\mathcal{K}$ function $\mu$ such that if $||\xv_0||  \leq a \Rightarrow  ||\xv(t)||\leq Le^{-\beta t}||\xv_0||+ \mu(||\dv||_\infty ),\, \forall t \geq 0$; where $||\cdot||_\infty$  is the maximum norm.
Since $\beta>0$, solutions converge to a neighborhood of the origin which depends on the size of the disturbance. Exponential ISS is achieved by requiring that
\begin{equation}\label{eq:iss}
\dot{V}(\xv,\uv,\dv) \leq -\lambda V(\xv)+\iota( ||\dv||_\infty),
\end{equation} for some class-$\mathcal{K}$ function $\iota$. Similarly, safety can be extended to \textit{input-to-state safety} (ISSf) by requiring that the system stays within a neighbourhood $\mathcal{S}_d \supseteq  \mathcal{S}$ of the safe set $\mathcal{S}$, which clearly depends on the entity of the disturbance.
% , i.e. $\xv_0 \in \mathcal{S}_d \Rightarrow \xv(t) \in \mathcal{S}_d, \forall t\geq 0$. 
This neighbourhood is defined as a 0-superlevel set
\begin{equation}
    S_d = \{\xv \in X: h(\xv) + \gamma(||\dv||_\infty)\geq 0\},
\end{equation}
with some class-$\mathcal{K}$ function $\gamma$. ISSf is guaranteed by imposing over \eqref{cond:2} in Theorem \ref{th:cbf}
\begin{equation}
    \dot{h}(\xv,\uv,\dv)\geq -\alpha h(\xv) - \iota(||\dv||_\infty).
\end{equation}